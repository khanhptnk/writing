\documentclass[11pt,letterpaper]{article}
\usepackage[top=1in,bottom=1in,left=1in,right=1in]{geometry}
\usepackage[numbers]{natbib}      % http://merkel.zoneo.net/Latex/natbib.php
\usepackage{palatino}
\bibpunct{(}{)}{;}{a}{,}{,}
\usepackage{chngpage}
\usepackage{stmaryrd}
\usepackage{amssymb}
\usepackage{amsmath}
\usepackage{amsthm}
\usepackage{graphicx}
\usepackage{lscape}
\usepackage{subfigure}
\usepackage{parskip}
\usepackage[usenames,dvipsnames]{color}
\usepackage{indentfirst}
\definecolor{myblue}{rgb}{0,0.1,0.6}
\definecolor{mygreen}{rgb}{0,0.3,0.1}
\usepackage[colorlinks=true,linkcolor=black,citecolor=mygreen,urlcolor=myblue]{hyperref}
\newcommand{\bocomment}[1]{\textcolor{Bittersweet}{[#1 -BTO]}}
\newenvironment{itemizesquish}{\begin{list}{\labelitemi}{\setlength{\parskip}{0.6cm}\setlength{\itemsep}{0em}\setlength{\labelwidth}{2em}\setlength{\leftmargin}{\labelwidth}\addtolength{\leftmargin}{\labelsep}}}{\end{list}}
\newcommand{\norm}[1]{\left\lVert#1\right\rVert}
\newcommand{\ignore}[1]{}

\newcommand{\solution}[1]{{\color{Blue}[\textbf{Solution:} #1]}}
\theoremstyle{definition}
\newtheorem{question}{Question}[section]
% \newtheorem{question}{Question}

\bibliographystyle{plainnat}
\setlength{\parindent}{30pt}
\linespread{1}

\title{
    Linear Regression and Logistic Regression  
}

\author{
	Khanh Nguyen
}

\date{June 02, 2015}

\begin{document}
\maketitle

We consider data with response $y$ and input $x = \{x_1, x_2, ..., x_D\}$. We model the relationship between $x$ and $y$ by a probabilistic model of $p(y | x)$.

Linear regression and logistic regression are examples of parametric models, i.e. models that have fixed amounts of parameters. We denote by $w$ the model parameter.

\section{Linear Regression}

  In linear regression, $y$ is a linear function of $x$:

\begin{equation}
y(x) = w^{T}x + \epsilon = \sum_{j=1}^D w_jx_j + \epsilon
\end{equation}
\label{eqn:lr}

We assume that $\epsilon$ has a Gaussian distribution, i.e. $\epsilon \sim \mathcal{N}(w_0, \sigma^2)$. We can rewrite the linear regression model as a conditional probability density:

\begin{equation}
p(y | x, \theta) = \mathcal{N}(y | w^{T}x, \sigma^2)
\end{equation}

where $\theta = (w, \sigma^2)$ \footnote{Note that here $x$ and $w$ are slightly different from Eqn. \ref{eqn:lr}: $x = \{1, x1, ..., x_D\}$ and $w = \{w_0, w_1, .., w_D\}$}. $\mathcal{N}(y | w^{T}x, \sigma^2)$ is a normal distribution with mean depending on $x$. To represent non-linear relationship between $x$ and $y$, we can replace the mean and variance of the distribution by any non-linear function of $x$.

\section{Logistic Regression}

Despite of the name, logistic regression is used for classification. $y$ is now a binary variable, taking value of either 0 or 1. Hence, we need a different distribution than Gaussian to model this fact. Bernoulli distribution is a natural choice:

\begin{equation}
p(y | x, w) = Ber(y | \mu(x))
\end{equation}

where $\mu(x)$ is the parameter for the Bernoulli distribution and is a function of $x$.

Since $\mu(x)$ must be between 0 and 1, we cannot use $w^{T}x$ but have to transform it somehow to fit into the [0,1] interval. We use the \emph{sigmoid function}, which is defined as:

\begin{equation}
sigm(t) = \frac{1}{1 + e^{-t}} = \frac{e^t}{1 + e^t}
\end{equation} 

Setting $\mu(x) = sigm(x)$, we obtain the logistic regression model:

\begin{equation}
p(y | x, w) = Ber(y | sigm(x))
\end{equation}

\end{document}






